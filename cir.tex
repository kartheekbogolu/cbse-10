
\documentclass{article}
\newcommand\tab[1][1cm]{\hspace*{#1}}
\usepackage{graphicx}
\usepackage{multicol}
\begin{document}
\begin{enumerate}
\item In the given figure, PQ is tangent to the circle centred at O.If $\angle{AOB}$=95$^\circ$, then the measure of $\angle{ABQ}$ will be
	\begin{figure}[h]
		\centering
	\includegraphics[width=0.35\textwidth]{a2}
\end{figure}
		\begin{multicols}{2}
			\begin{enumerate}
		\item 47.5$^\circ$
		\item 42.5$^\circ$
		\item 85$^\circ$
		\item 95$^\circ$
			\end{enumerate}
		\end{multicols}
	\item (a) Two tangents TP and TQ are drawn to a circle with center O from an external point T. prove that $\angle{PTQ}$=2$\angle{OPQ}$
		\begin{figure}[h]
			\centering
			\includegraphics[width=0.35\textwidth]{cir1}
		\end{figure}
		$$\textbf{OR}$$
		(b) In the given figure, a circle is inscribed in a quadrilaterals ABCD in which $\angle{B}$=90$^\circ$. If AD=7 cm,AB=20 cm and DS=3 cm, then find the radius of the circle
		\begin{figure}[h]
			\centering
			\includegraphics[width=0.335\textwidth]{cir2}
		\end{figure}
	\item The discus throw is an event in which an athlete attempts tothrow a disus.the athlete spins anti-clockwise around one and a half times through a circle, then the release throw. when released,the discus travls along tangnt to the circar spin orbit.
		\begin{figure}[h]
			\centering
			\includegraphics[width=0.35\textwidth]{cir3}
		\end{figure}\\
		In the given figure, AB is one such tangent to a circle of radius 75 cm.Point O is cnter of the circe and $\angle{ABO}$=30$^\circ$. PQ is paralel to OA
		\begin{figure}[h]
			\centering
			\includegraphics[width=0.35\textwidth]{ac}
		\end{figure}\\
		Based on above informtion:\\
		(a) find the length of AB.\\
		(b) find the length of OB.\\
		(c) find the length of PQ.\\
		\tab[2cm]OR\\
		\tab[0.5cm]find the length of PQ.
	\item In the given figure, the quadriateral PQRS circumscribes a circle. Here PA+CS is equal to:
		\begin{figure}[h]
			\centering
			\includegraphics[width=0.35\textwidth]{cir5.1}
		\end{figure}
		\begin{multicols}{2}
			\begin{enumerate}
				\item QR
				\item PS
				\item PR
				\item PQ
			\end{enumerate}
		\end{multicols}
	\item In the given figure, O is the center of the circle. AB and AC are tangents drawn to the circle from point A.If $\angle{BAC}$=65$^\circ$, then find the measure of $\angle{BOC}$.
		\begin{figure}[h]
			\centering
			\includegraphics[width=0.35\textwidth]{ab}
		\end{figure}
	\item In the given figure, O is the center of the circle and QPR is a tangent to it at P. prove that $\angle{QAP}$+$\angle{APR}$=90$^\circ$
		\begin{figure}[h]
			\centering
			\includegraphics[width=0.35\textwidth]{ad}
		\end{figure}
	\item In the givien figure, PT is a tangent at T to the circle  with center O.If $\angle{TPO}$=25$^\circ$, then x is equal to:
		\begin{figure}[h]
			\centering
			\includegraphics[width=0.4\textwidth]{a1}
		\end{figure}
		\begin{enumerate}
			\item 25$^\circ$
			\item 65$^\circ$
			\item 90$^\circ$
			\item 115$^\circ$
		\end{enumerate}
	\item In the given, TA is a tangent to the circle with center O such that OT=4cm, $\angle${OTA}=30$^\circ$, then length of TA is:
		\begin{figure}[h]
			\centering
			\includegraphics[width=0.6\textwidth]{ae}
		\end{figure}
		\begin{enumerate}
			\item 2$\sqrt{3}$cm
			\item 2cm
			\item 2$\sqrt{2}$cm
			\item $\sqrt{3}$cm
		\end{enumerate}
	\item Two concentric circles are of radii 5cm and 3cm. Find the length of the chord of the larger circle hich touches the smaller circle



\end{enumerate}
\end{document}
