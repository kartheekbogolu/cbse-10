\documentclass[10pt,a4paper]{article}
\usepackage[a4paper,outer=1.5cm,inner=1.5cm,top=1.75cm,bottom=1.5cm]{geometry}
\usepackage{graphicx}
\usepackage{multicol}
\usepackage{tabularx}
\usepackage{amsmath}
\begin{document}
\centerline{\textbf{IDE ASSIGNMENT}}
\centerline{Bogolu Kartheek}
\centerline{bogolukarthik@gmail.com}
\centerline{FWC22136 IITH - Future Wireless Communications}
\graphicspath{{./Documents}{./figs}}
\tableofcontents
\section{Problem}
(GATE2019-QP-EE)\\
Q.35 The output expression for the Karnaugh map shown below is
\begin{figure}[h]
	\centering
	\includegraphics[width=0.5\columnwidth]{35.jpg}
	\caption{K-MAP}
	\label{fig=pic}
\end{figure}
\begin{multicols}{2}
\begin{enumerate}
	\item QR'+S
	\item QR+S
		\item QR'+S'
		\item QR+S'
\end{enumerate}
\end{multicols}
\section{Components}
\begin{tabularx}{0.8\textwidth}{
		| >{\centering\arraybackslash}X
		| >{\centering\arraybackslash}X
		| >{\centering\arraybackslash}X |}
	\hline
	 Components &  Value & Quantity \\
	\hline
	 Breadboard &  & 1 \\
	 \hline
	 Arduino & uno & 1 \\
	 \hline
	 Jumper wires &  & 4 \\
	 \hline
\end{tabularx}
\subsection{Arduino}
The Arduino Uno has some ground pins.analog input 
pins A0-A3 and digial pins D1-D13 that can be used
for both input as well as output. It also has two
powe pins that can generate 3.3V and 5V. In the
following exercise,we use digital pins,GND and 5V
\section{Implementation}
\subsection{Truth table}
\begin{tabularx}{0.8\textwidth}{
		| >{\centering\arraybackslash}X
		| >{\centering\arraybackslash}X
		| >{\centering\arraybackslash}X | }
	\hline
	A & B & X=A'+B'\\
	\hline
	0 & 0 & 1 \\
	\hline
	0 & 1 & 0 \\
	\hline
	1 & 0 & 0 \\
	\hline
	1 & 1 & 0 \\
	\hline
\end{tabularx}
\subsection{K-MAP}
\begin{figure}[h]
	\centering
	\includegraphics[width=0.5\columnwidth]{36.jpg}\\
	\caption{K-MAP}
	\label{fig:pic2}
\end{figure}
\subsection{Boolean Equation}
\begin{align}
	F&=R'SP'(Q'+Q)+R'SP(Q'+Q)+R'S'Q(P'+P)+R'SQ(P'+P)+RSP'(Q'+Q)+RSP(Q'+Q) \nonumber\\
	F&=R'S(P'+P)+R'Q(S'+S)+RS(P'+P) \nonumber\\
	F&=S(R'+R)+R'Q \nonumber\\
	F&=S+QR'\nonumber\\
\end{align}
\section{Hardware}
\begin{enumerate}
	\item Connect one end of jumper wire to the ground pin on the Arduino no and other end to the 
		breadboard's ground rail(-)
	\item Connect the one terminal of jumper wire to the input pins of Arduino and other end to the positive
		rail(+) on the breadboard
	\item Connect one end of another jumper wire to the inpur pin of Arduino and other end to the positive
		to rail(+)
	\item Enable the power supply to breadboard from arduino by connecting one end of jumper wire to the 
		power pin of arduino and other end to the positive rail on the breadboard
\end{enumerate}
\section{Conclusion}
Hence we have implemented the NOR gate by the code-givien below \\
\framebox{https://github.com/kartheekbogolu/cbse-10/tree/main/codes}
\end{document}
